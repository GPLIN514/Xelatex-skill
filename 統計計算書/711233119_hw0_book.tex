\addcontentsline{toc}{chapter}{序}
\chapter*{緒論}
研究生階段是學術生涯中的一個關鍵階段,而論文的撰寫則是評估學生研究成果的重要指標。隨著科技進步,傳統的文書處理工具逐漸被專業排版軟體 LaTeX 取代,這其中有一項顯著的優勢,即 LaTeX 具備豐富的數學式子展現功能,使其成為統計研究所學生不可或缺的利器。本書的焦點之一即是介紹 LaTeX 的運用,特別針對數學式子的排版,使讀者能夠在論文中以專業且美觀的方式呈現數學概念。

在現代數據科學的潮流中,Python 被廣泛應用於數據分析、機器學習和深度學習等領域。本書透過 Python 的實際示範,將讀者帶入探索機率分配的世界。通過對各種分配的特性深入剖析,讀者不僅能夠理解分配的形態和變化趨勢,更能夠在未來的研究中靈活運用這些分配模型。緊接著,本書關注於迴歸模型的應用,並從實際生成的資料中進行探討。這不僅讓讀者瞭解迴歸模型在真實場景中的效能,同時提供了一個實用的操作框架,使讀者能夠將所學應用於自身的研究議題中。

在學術研究中,機器學習和深度學習等領域的興起不可忽視。本書進一步介紹了線性判別分析(LDA)、二次判別分析(QDA)、K最近鄰(KNN)和人工神經網絡(ANN)等學習器的應用,並針對這些方法在實際問題中的表現進行了詳盡的探討。這使讀者能夠全面了解不同學習器之間的優勢和限制,為未來的研究方向提供更多的參考和思考。

總體來說,本書以實用性為出發點,旨在幫助研究生更好地應對論文撰寫和數據科學的挑戰。透過 LaTeX 和 Python 這兩項重要工具的深度結合,讀者將獲得更廣泛的學術技能,為未來的學術和職業發展打下穩固基礎。