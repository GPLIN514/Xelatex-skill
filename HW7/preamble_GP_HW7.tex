%\documentclass[12pt, a4paper]{article} 

\usepackage{fontspec} % Font selection for XeLaTeX; see fontspec.pdf. 
\usepackage{xeCJK}	% 中文使用 XeCJK,利用 \setCJKmainfont 定義中文內文、粗體與斜體的字型
\defaultfontfeatures{Mapping=tex-text} % to support TeX conventions like ``---''
\usepackage{xunicode} % Unicode support for LaTeX character names(accents, European chars, etc)
\usepackage{xltxtra} 				% Extra customizations for XeLaTeX
\usepackage{amsmath, amssymb, amsfonts}
\usepackage{enumerate}
\usepackage{graphicx,subfig,float,wrapfig} % support the \includegraphics command and options
\usepackage[outercaption]{sidecap} %[options]=[outercaption], [innercaption], [leftcaption], [rightcaption]
\usepackage{array, booktabs}
\usepackage{color, xcolor}
\usepackage{longtable}
\usepackage{colortbl}
\usepackage{amscd}
\usepackage{diagbox}
\usepackage{tikz}
\usetikzlibrary{matrix} 
\usepackage[utf8]{inputenc}
\usepackage[english]{babel}
\usetikzlibrary{positioning}                         			\usepackage{arydshln}
\usepackage{wasysym}
\usepackage{nicematrix}
\usepackage{hyperref}
\usepackage{listings}						% 直接將 latex 碼轉換成顯示文字
\usepackage[parfill]{parskip} 				% 新段落前加一空行,不使用縮排
\usepackage[left=1.5in,right=1in,top=1in,bottom=1in]{geometry} %若要製作投影片則不需此行
\usepackage{url}
%-----------------------------------------------------------------
%  中英文內文字型設定
\setCJKmainfont							% 設定中文內文字型
	[
%		BoldFont=Microsoft YaHei	    %定義粗體的字型(Win)
		BoldFont=蘋方-繁 中粗體    		%定義粗體的字型(Mac)
	]
%	{新細明體}						% 設定中文內文字型(Win)
	{宋體-繁 標準體}							% 設定中文內文字型(Mac)	
\setmainfont{Times New Roman}		% 設定英文內文字型
\setsansfont{Arial}					% 無襯字字型 used with {\sffamily ...}
%\setsansfont[Scale=MatchLowercase,Mapping=tex-text]{Gill Sans}
\setmonofont{Courier New}			% 等寬字型 used with {\ttfamily ...}
%\setmonofont[Scale=MatchLowercase]{Andale Mono}
% 其他字型(隨使用的電腦安裝的字型不同,用註解的方式調整(打開或關閉))
% 英文字型
\newfontfamily{\E}{Didot}				
\newfontfamily{\A}{Arial}
\newfontfamily{\C}[Scale=0.9]{Arial}
\newfontfamily{\R}{Times New Roman}
\newfontfamily{\TT}[Scale=0.8]{Times New Roman}
\newfontfamily{\RR}{Helvetica Neue}
\newfontfamily{\Geo}{Georgia}
% 中文字型
\newCJKfontfamily{\MB}{黑體-繁}	% 等寬及無襯線字體 Mac
\newCJKfontfamily{\SM}[Scale=0.8]{宋體-繁}	% 縮小版(Mac)
\newCJKfontfamily{\PP}{翩翩體-繁}        	
\newCJKfontfamily{\BB}{蘋方-繁}		    % 粗體 Mac
\newCJKfontfamily{\WW}{娃娃體-繁}
\newCJKfontfamily{\WB}{魏碑-繁}            %粗體
\newCJKfontfamily{\KT}{楷體-繁}            % Mac下的標楷體
\newCJKfontfamily{\YT}{圓體-繁}            %粗體
\newCJKfontfamily{\LS}{儷宋Pro}
\newCJKfontfamily{\LH}{凌慧體-繁}          %手寫
\newCJKfontfamily{\YP}{雅痞-繁}          %手寫
\newCJKfontfamily{\SK}{行楷-繁}            %細體
%\newCJKfontfamily{\SS}{蘋果儷細宋}          %細體
% 以下為自行安裝的字型:CwTex 組合
%\newCJKfontfamily{\CF}{cwTeX Q Fangsong Medium}	% CwTex 仿宋體
%\newCJKfontfamily{\CB}{cwTeX Q Hei Bold}			% CwTex 粗黑體
%\newCJKfontfamily{\CK}{cwTeX Q Kai Medium}   	% CwTex 楷體
%\newCJKfontfamily{\CM}{cwTeX Q Ming Medium}		% CwTex 明體
%\newCJKfontfamily{\CR}{cwTeX Q Yuan Medium}		% CwTex 圓體
%-----------------------------------------------------------------------------------------------------------------------
\XeTeXlinebreaklocale "zh"             		%這兩行一定要加,中文才能自動換行
\XeTeXlinebreakskip = 0pt plus 1pt     		%這兩行一定要加,中文才能自動換行
%-----------------------------------------------------------------------------------------------------------------------
\newcommand{\cw}{\texttt{cw}\kern-.6pt\TeX}	% 這是 cwTex 的 logo 文字			
\newcommand{\imgdir}{./images/}% 設定圖檔的目錄位置
\renewcommand{\tablename}{表}	% 改變表格標號文字為中文的「表」(預設為 Table)
\renewcommand{\figurename}{圖}% 改變圖片標號文字為中文的「圖」(預設為 Figure)

% 設定顏色 see color Table: http://latexcolor.com
\definecolor{slight}{gray}{0.9}				
\definecolor{airforceblue}{rgb}{0.36, 0.54, 0.66} 
\definecolor{arylideyellow}{rgb}{0.91, 0.84, 0.42}
\definecolor{babyblue}{rgb}{0.54, 0.81, 0.94}
\definecolor{cadmiumred}{rgb}{0.89, 0.0, 0.13}
\definecolor{coolblack}{rgb}{0.0, 0.18, 0.39}
\definecolor{beaublue}{rgb}{0.74, 0.83, 0.9}
\definecolor{beige}{rgb}{0.96, 0.96, 0.86}
\definecolor{bisque}{rgb}{1.0, 0.89, 0.77}
\definecolor{gray(x11gray)}{rgb}{0.75, 0.75, 0.75}
\definecolor{limegreen}{rgb}{0.2, 0.8, 0.2}
\definecolor{splashedwhite}{rgb}{1.0, 0.99, 1.0}
\definecolor{usccardinal}{rgb}{0.6, 0.0, 0.0}%咖
\definecolor{violet}{rgb}{0.56, 0.0, 1.0}%亮紫
\definecolor{blue-violet}{rgb}{0.54, 0.17, 0.89}%淺紫
\definecolor{navyblue}{rgb}{0.0, 0.0, 0.5}%海軍藍
\definecolor{mediumblue}{rgb}{0.0, 0.0, 0.8}%中藍
\definecolor{blue}{rgb}{0.0, 0.0, 1.0}%藍
\definecolor{deepskyblue}{rgb}{0.0, 0.75, 1.0}%深天藍
\definecolor{electriccyan}{rgb}{0.0, 1.0, 1.0}%藍綠
\definecolor{caribbeangreen}{rgb}{0.0, 0.8, 0.6}%加勒比綠
\definecolor{lime(web)(x11green)}{rgb}{0.0, 1.0, 0.0}%螢光綠
\definecolor{springgreen}{rgb}{0.0, 1.0, 0.5}%春綠
\definecolor{magicmint}{rgb}{0.67, 0.94, 0.82}%淺青綠
\definecolor{fluorescentyellow}{rgb}{0.8, 1.0, 0.0}%黃綠
\definecolor{magicmint}{rgb}{0.67, 0.94, 0.82}
\definecolor{bubbles}{rgb}{0.91, 1.0, 1.0} %極淡綠
\definecolor{cream}{rgb}{1.0, 0.99, 0.82} %極淡黃
\definecolor{electricyellow}{rgb}{1.0, 1.0, 0.0}%亮黃
\definecolor{sandstorm}{rgb}{0.93, 0.84, 0.25}%暗黃
\definecolor{citrine}{rgb}{0.89, 0.82, 0.04}%深黃
\definecolor{amber(sae/ece)}{rgb}{1.0, 0.49, 0.0}%橘
\definecolor{ruddy}{rgb}{1.0, 0.0, 0.16}%紅
\definecolor{razzledazzlerose}{rgb}{1.0, 0.2, 0.8}%粉
\definecolor{lavenderpink}{rgb}{0.98, 0.68, 0.82}%淺粉
\definecolor{mistyrose}{rgb}{1.0, 0.89, 0.88}
\definecolor{burlywood}{rgb}{0.87, 0.72, 0.53}%淺木頭
%---------------------------------------------------------------------
\hypersetup
{
      pdfauthor=Unnamed0range,
      colorlinks=true,
      linkcolor=black,
      anchorcolor=black,
      citecolor=black,
      urlcolor=black
}
%---------------------------------------------------------------------
% 映出程式碼 \begin{lstlisting} 的內部設定
\lstset
{	language=[LaTeX]TeX,
    breaklines=true,
    %basicstyle=\tt\scriptsize,
    basicstyle=\R\normalsize,
    keywordstyle=\color{black},
    identifierstyle=\color{black},
    commentstyle=\color{mediumblue},
    stringstyle=\rmfamily,%\itshape 英文字會變斜體
    showstringspaces=false,
    %backgroundcolor=\color{splashedwhite},
    backgroundcolor=\color{slight},
    frame=single,							%default frame=none 
    rulecolor=\color{gray(x11gray)},
    framerule=0.4pt,							%expand outward 
    framesep=3pt,							%expand outward
    xleftmargin=3.4pt,		%to make the frame fits in the text area. 
    xrightmargin=3.4pt,		%to make the frame fits in the text area. 
    tabsize=2				%default :8 only influence the lstlisting and lstinline.
}

% 映出程式碼 \begin{lstlisting} 的內部設定 for Python codes
%\lstset{language=Python}
%\lstset{frame=lines}
%\lstset{basicstyle=\SCP\normalsize}
%\lstset{keywordstyle=\color{blue}}
%\lstset{commentstyle=\color{airforceblue}\itshape}
%\lstset{backgroundcolor=\color{beige}}