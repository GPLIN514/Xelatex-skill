\documentclass[xcolor=dvipsnames]{beamer}
\usecolortheme[named=Blue]{structure}
\input{preamble_GP_hw1.tex}  % 使用自己維護的定義檔
\usetheme{Warsaw}
\useoutertheme{miniframes}
\title{ Blockchain}
\author{{\KT 林貫原}}
\date{{\R \today }}

\begin{document}

\maketitle
\begin{frame}{\textbf{目錄}}
	\tableofcontents % 將所有節及子節的內容輸出成表
\end{frame}
\section{何謂區塊鏈}
\begin{frame}
\frametitle{何謂區塊鏈}
\begin{center}\colorbox{slight}{\begin{tabular}{p{0.9\textwidth}}
    {\KT 區塊鏈並非比特幣}
\end{tabular}}
\end{center}
\pause
\begin{center}\colorbox{slight}{\begin{tabular}{p{0.9\textwidth}}
    {\KT 區塊鏈是一種透過共識演算法實現信任去中心化的技術}
\end{tabular}}
\end{center}
\pause
\begin{itemize}
    \item 去中心化
    \item 匿名
    \item 不可篡改
    \item 加密
\end{itemize}
\end{frame}
\begin{frame}
\frametitle{何謂區塊鏈}
\begin{figure}
    \centering{
        \includegraphics[scale=0.2]{\imgdir 區塊鏈.png}}
    \label{fig:blockchain}
    \end{figure}
\end{frame}
\begin{frame}
\frametitle{何謂區塊鏈}
\begin{figure}
    \centering{
        \includegraphics[scale=0.15]{\imgdir 區塊鏈1.png}}
\end{figure}
    \pause
\begin{figure}
    \centering{
        \includegraphics[scale=0.2]{\imgdir 區塊鏈2.png}}
\end{figure}
\end{frame}
\begin{frame}
\frametitle{以太坊-燃料}
交易手續費
\pause
\begin{columns}
    \begin{column}{0.5\textwidth} % 左側列
\begin{block}{
\begin{itemize}
    \item Gas Limit
    \item Gas Price
    \item Gwei
    \item Wei
\end{itemize}}
\end{block}
    \end{column}
    \pause
    \begin{column}{0.5\textwidth} % 右側列
\begin{itemize}
    \item $1Ether=10^{18}Wei$
    \item $1Gwei=10^9Wei$
\end{itemize}
    \end{column}
\end{columns}
\pause
\begin{figure}
    \centering{
        \includegraphics[scale=0.65]{\imgdir gas計算.png}}
    \label{fig:gas計算}
    \end{figure}
\end{frame}
\begin{frame}
\frametitle{以太坊-燃料}
\begin{figure}
    \centering{
        \includegraphics[scale=0.5]{\imgdir Gas1.png}}
    \label{fig:gas1}
    \end{figure}
\pause
\begin{figure}
    \centering{
        \includegraphics[scale=0.47]{\imgdir Gas2.png}}
    \label{fig:gas2}
    \end{figure}
\end{frame}

\section{Decentralized}

\begin{frame}
\frametitle{Decentralized}
\begin{columns}
    \begin{column}{0.5\textwidth} 
      \includegraphics[width=\textwidth]{\imgdir centralization.png}
    \end{column}
    \pause
    \begin{column}{0.5\textwidth} 
      \includegraphics[width=\textwidth]{\imgdir decentralization.png}
    \end{column}
  \end{columns}
\end{frame}
\begin{frame}
\frametitle{分叉}
\pause
\begin{columns}
    \begin{column}{0.5\textwidth} % 左側列
\begin{block}{
\begin{itemize}
    \item 工作狀態不一致
    \item GHOST協議
\end{itemize}}
\end{block}
    \end{column}
    \pause
    \begin{column}{0.5\textwidth} % 右側列
\begin{itemize}
    \item 軟分叉:網路延遲
    \item 硬分叉:共識機制改變
\end{itemize}
    \end{column}
\end{columns}
\pause
\begin{figure}
    \centering{
        \includegraphics[scale=0.5]{\imgdir Forks.png}}
    \label{fig:Forks}
    \end{figure}
\end{frame}

\section{Etherscan}

\begin{frame}
\frametitle{Etherscan}
以太坊瀏覽器
    \pause
\begin{columns}
    \begin{column}{0.3\textwidth} % 左側列
\begin{itemize}
    \item 瀏覽及查詢
    \pause
    \item 不可交易
    \pause
    \item 不可存私鑰
    \pause
    \item 不一定要註冊
    \pause
\end{itemize}
    \end{column}
    \begin{column}{0.7\textwidth} % 右側列
      \includegraphics[width=\textwidth]{\imgdir Etherscan.png}
    \end{column}
  \end{columns}
\end{frame}

\section{Ethereum}

\begin{frame}
\frametitle{以太坊-私鑰與公鑰}
\begin{columns}
    \begin{column}{0.35\textwidth} % 左側列
\begin{block}{\begin{itemize}
    \item 私鑰(256bits)
    \pause
    \begin{itemize}
        \item 簽署交易$\rightarrow$ 對你的帳戶資產進行更改
        \item 密碼學特性
        \item 僅自己所有
        \item 將訊息簽名
    \end{itemize}
    \pause
\end{itemize}}
\end{block}
    \end{column}
    \begin{column}{0.65\textwidth} % 右側列
      \includegraphics[width=\textwidth]{\imgdir 單向函數.png}
    \end{column}
\end{columns}
\end{frame}
\begin{frame}
\frametitle{以太坊-公鑰與地址}
\begin{columns}
    \begin{column}{0.5\textwidth} % 左側列
\begin{block}{
\begin{itemize}
    \item 公鑰(512bits)
    \begin{itemize}
        \item 私鑰$\rightarrow$公鑰(ECDS)
        \item 驗證簽名
    \end{itemize}
\end{itemize}}
\end{block}
    \end{column}
    \pause
    \begin{column}{0.5\textwidth} % 右側列
\begin{block}{
\begin{itemize}
    \item 地址(160bits)
    \begin{itemize}
        \item 公鑰$\rightarrow$地址(hash)
        \item 交易發起地
        \item 以太坊帳戶
    \end{itemize}
\end{itemize}}
\end{block}
    \end{column}
\end{columns}
\end{frame}
\begin{frame}
\frametitle{以太坊-私鑰公鑰地址}
\begin{figure}
    \centering{
        \includegraphics[scale=0.445]{\imgdir 私鑰公鑰地址.png}}
    \end{figure}
\end{frame}
\begin{frame}
\frametitle{以太坊-私鑰公鑰地址}
\begin{figure}
    \centering{
        \includegraphics[scale=0.45]{\imgdir 公私鑰示意圖1.png}}
    \end{figure}
\end{frame}
\begin{frame}
\frametitle{以太坊-私鑰公鑰地址}
\begin{figure}
    \centering{
        \includegraphics[scale=0.5]{\imgdir 公私鑰示意圖2.png}}
    \end{figure}
\end{frame}
\begin{frame}
\frametitle{雜湊函數}
Hash Function 是一種數學函數,它將任意長度的輸入,轉換成相同長度的輸出,其輸出叫做雜湊值(hash value) 。以下是他的四大特性:
    \pause
\hyperlink{https://emn178.github.io/online-tools/sha256.html}{\beamerbutton{跳}}    
\begin{itemize}
    \item 固定輸出長度
    \pause
    \item 唯一性
    \pause
    \item 效能高
    \pause
    \item 不可逆
\end{itemize}
\end{frame}

\section{BTC V.S. ETH}

\begin{frame}
\frametitle{BTC V.S. ETH}
\begin{columns}
    \begin{column}{0.5\textwidth} % 左側列
    \begin{block}{
\begin{itemize}
    \item 比特幣
    \begin{itemize}
        \item SHA-256
    \pause
        \item 雙重雜湊
    \pause
        \item 每秒3-4筆交易
    \pause
        \item 只具有支付功能
    \end{itemize}
\end{itemize}}
\end{block}
    \end{column}
    \pause
    \begin{column}{0.5\textwidth} % 右側列
    \begin{block}{
\begin{itemize}
    \item 以太幣
    \pause
    \begin{itemize}
        \item Keccak(SHA-3)
    \pause
        \item 密碼學特性
    \pause
        \item 每秒20筆交易
    \pause
        \item 執行智能合約
    \end{itemize}
\end{itemize}}
\end{block}
    \end{column}
\end{columns}
\end{frame}

\section{Metamask}

\begin{frame}
\frametitle{Metamask}
\begin{itemize}
        \item 是加密貨幣領域中廣泛使用的錢包之一
        \item 作為瀏覽器擴展程式
\end{itemize}
\begin{figure}
    \centering{
        \includegraphics[scale=0.6]{\imgdir matamask.png}}
    \end{figure}
    \end{frame}
    
\section{Smart Contract}
    
\begin{frame}
\frametitle{Smart Contract}
目標:更安全成本更低
    \pause
\begin{itemize}
        \item 自動執行
    \pause
        \item 不可逆轉
    \pause
        \item 去中心化
    \pause
        \item 可程式化
\end{itemize}
\end{frame}

\section{Solidity \& Remix}

\begin{frame}
\frametitle{Solidity \& Remix}
\begin{columns}
    \begin{column}{0.5\textwidth} % 左側列
\begin{block}{
\begin{itemize}
    \item Solidity
    \pause
    \begin{itemize}
        \item 主要流行於以太坊
        \item 編寫智能合約的語言
        \item 靜態語言
    \end{itemize}
\end{itemize}}
\end{block}
    \end{column}
    \begin{column}{0.5\textwidth} % 右側列
\begin{block}{
\begin{itemize}
    \item Remix
    \pause
    \begin{itemize}
        \item 以太坊提供的線上版官方整合開發環境(IDE)
        \item 兼具編譯,部署,測試等功能
    \end{itemize}
\end{itemize}}
\end{block}
    \end{column}
\end{columns}
\end{frame}
\begin{frame}
\frametitle{Remix}
\begin{figure}
    \centering{
        \includegraphics[scale=0.21]{\imgdir remix.png}}
    \end{figure}
\end{frame}
\begin{frame}[plain] % 產生無外框的頁面
	\begin{center}
		{\Huge Thank you for listening}
	\end{center}
\end{frame}
\end{document}




