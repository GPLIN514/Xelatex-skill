\usepackage{fontspec}
\usepackage{xeCJK}
\defaultfontfeatures{Mapping=tex-text}
\usepackage{xunicode}
\usepackage{xltxtra}
\usepackage{amsmath, amssymb}
\usepackage{enumerate}
\usepackage{graphicx,subfig,float,wrapfig}
\usepackage[outercaption]{sidecap}
\usepackage{array, booktabs}
\usepackage{color, xcolor}
\usepackage{longtable}
\usepackage{colortbl}      				
\usepackage{listings}
\usepackage[parfill]{parskip}
\usepackage[left=1.5in,right=1in,top=1in,bottom=1in]{geometry}  %% 建議不要用!!(通常是為了裝訂)
\usepackage{url}
\usepackage{gensymb} % 可以用度C
\usepackage{multirow}
\usepackage{arydshln}
% [裡面放設定]{放套件}
\usepackage{tikz}
\usetikzlibrary{shapes.geometric}
\usepackage{pgf-pie}
%\usepackage{subfigure}
%\usepackage{subcaption}
\usepackage{bm}

% 內文(中文、英文字型)----------------------------------------------------
\setCJKmainfont[BoldFont=Microsoft YaHei]{新細明體}	    % 粗體字型、設定中文內文字型
\setmainfont{Times New Roman}		% 設定英文內文字型
\setsansfont{Arial}					% 無襯字型
\setmonofont{Courier New}			% 等寬字型

% 英文字型
\newfontfamily{\E}{Calibri}				
\newfontfamily{\A}{Arial}
\newfontfamily{\C}[Scale=0.9]{Arial}
\newfontfamily{\R}{Times New Roman}
\newfontfamily{\TT}[Scale=0.8]{Times New Roman}

% 中文字型
\newCJKfontfamily{\MB}{微軟正黑體}			% 等寬及無襯線字體 Win
\newCJKfontfamily{\SM}[Scale=0.8]{新細明體}	% 縮小版(Win)
\newCJKfontfamily{\K}{標楷體}                	% Windows下的標楷體
\newCJKfontfamily{\BB}{Microsoft YaHei}		% 粗體 Win

% 以下為自行安裝的字型:CwTex 組合
%\newCJKfontfamily{\CF}{cwTeX Q Fangsong Medium}	% CwTex 仿宋體
%\newCJKfontfamily{\CB}{cwTeX Q Hei Bold}			% CwTex 粗黑體
%\newCJKfontfamily{\CK}{cwTeX Q Kai Medium}   	% CwTex 楷體
%\newCJKfontfamily{\CM}{cwTeX Q Ming Medium}		% CwTex 明體
%\newCJKfontfamily{\CR}{cwTeX Q Yuan Medium}		% CwTex 圓體

\XeTeXlinebreaklocale "zh"
\XeTeXlinebreakskip = 0pt plus 1pt

\newcommand{\cw}{\texttt{cw}\kern-.6pt\TeX}	% 這是 cwTex 的 logo 文字
\newcommand{\imgdir}{../work_1_images/}				% 設定圖檔的目錄位置
%% ../work_1_images/表示 : 在該檔案(..)的上一個目錄(/)的work_1_images(檔名work_1_images)裡面
\renewcommand{\tablename}{表}	% 改變表格標號文字為中文的「表」(預設為 Table)
\renewcommand{\figurename}{圖}% 改變圖片標號文字為中文的「圖」(預設為 Figure)

% 設定顏色 see color Table: http://latexcolor.com
\definecolor{slight}{gray}{0.9}				
\definecolor{airforceblue}{rgb}{0.36, 0.54, 0.66} 
\definecolor{arylideyellow}{rgb}{0.91, 0.84, 0.42}
\definecolor{babyblue}{rgb}{0.54, 0.81, 0.94}
\definecolor{cadmiumred}{rgb}{0.89, 0.0, 0.13}
\definecolor{coolblack}{rgb}{0.0, 0.18, 0.39}
\definecolor{beaublue}{rgb}{0.74, 0.83, 0.9}
\definecolor{beige}{rgb}{0.96, 0.96, 0.86}
\definecolor{bisque}{rgb}{1.0, 0.89, 0.77}
\definecolor{gray(x11gray)}{rgb}{0.75, 0.75, 0.75}
\definecolor{limegreen}{rgb}{0.2, 0.8, 0.2}
\definecolor{splashedwhite}{rgb}{1.0, 0.99, 1.0}

%我自己找的-------------------------------------------------
\definecolor{coolblue}{rgb}{0.0, 0.5, 1.0}
\definecolor{babypink}{rgb}{0.96, 0.76, 0.76}
\definecolor{beaublue}{rgb}{0.74, 0.83, 0.9}
\definecolor{bleudefrance}{rgb}{0.19, 0.55, 0.91}
\definecolor{burntorange}{rgb}{0.8, 0.33, 0.0}
\definecolor{cadetgrey}{rgb}{0.57, 0.64, 0.69}
\definecolor{camel}{rgb}{0.76, 0.6, 0.42}
\definecolor{ferngreen}{rgb}{0.31, 0.47, 0.26}
\definecolor{purpleheart}{rgb}{0.41, 0.21, 0.61}
\definecolor{peach-orange}{rgb}{1.0, 0.8, 0.6}
\definecolor{palecornflowerblue}{rgb}{0.67, 0.8, 0.94}
\definecolor{olivine}{rgb}{0.6, 0.73, 0.45}
\definecolor{navajowhite}{rgb}{1.0, 0.87, 0.68}
\definecolor{thistle}{rgb}{0.85, 0.75, 0.85}
\definecolor{titaniumyellow}{rgb}{0.93, 0.9, 0.0}
\definecolor{paletaupe}{rgb}{0.74, 0.6, 0.49}
\definecolor{blue(pigment)}{rgb}{0.2, 0.2, 0.6}  % 用在公式
%% 要打公式就\textcolor{blue(pigment)}{這邊打公式}
\definecolor{bostonuniversityred}{rgb}{0.8, 0.0, 0.0} % 用在答案
\definecolor{black}{rgb}{0.0, 0.0, 0.0}
\definecolor{cambridgeblue}{rgb}{0.64, 0.76, 0.68}
\definecolor{bananamania}{rgb}{0.98, 0.91, 0.71}
\definecolor{gray(x11gray)}{rgb}{0.75, 0.75, 0.75}
\definecolor{teagreen}{rgb}{0.82, 0.94, 0.75}
\definecolor{lightgreen}{rgb}{0.56, 0.93, 0.56}
\definecolor{honeydew}{rgb}{0.94, 1.0, 0.94}
\definecolor{babyblueeyes}{rgb}{0.63, 0.79, 0.95}
\definecolor{aliceblue}{rgb}{0.94, 0.97, 1.0}
\definecolor{carolinablue}{rgb}{0.6, 0.73, 0.89}
\definecolor{ceil}{rgb}{0.57, 0.63, 0.81}
\definecolor{columbiablue}{rgb}{0.61, 0.87, 1.0}
\definecolor{coolgrey}{rgb}{0.55, 0.57, 0.67}
\definecolor{lightgray}{rgb}{0.83, 0.83, 0.83}
\definecolor{dimgray}{rgb}{0.41, 0.41, 0.41}
\definecolor{gold(metallic)}{rgb}{0.83, 0.69, 0.22}
\definecolor{fuzzywuzzy}{rgb}{0.8, 0.4, 0.4}
\definecolor{bubblegum}{rgb}{0.99, 0.76, 0.8}
\definecolor{burgundy}{rgb}{0.5, 0.0, 0.13}

%---------------------------------------------------------------------
% 映出程式碼 \begin{lstlisting} 的內部設定
\lstset
{	language=[LaTeX]TeX,
    breaklines=true,
    %basicstyle=\tt\scriptsize,
    basicstyle=\tt\normalsize,
    keywordstyle=\color{blue},
    identifierstyle=\color{black},
    commentstyle=\color{limegreen}\itshape,
    stringstyle=\rmfamily,
    showstringspaces=false,
    %backgroundcolor=\color{splashedwhite},
    backgroundcolor=\color{slight},
    frame=single,							%default frame=none 
    rulecolor=\color{gray(x11gray)},
    framerule=0.4pt,							%expand outward 
    framesep=3pt,							%expand outward
    xleftmargin=3.4pt,		%to make the frame fits in the text area. 
    xrightmargin=3.4pt,		%to make the frame fits in the text area. 
    tabsize=2				%default :8 only influence the lstlisting and lstinline.
}

% 映出程式碼 \begin{lstlisting} 的內部設定 for Python codes
%\lstset{language=Python}
%\lstset{frame=lines}
%\lstset{basicstyle=\SCP\normalsize}
%\lstset{keywordstyle=\color{blue}}
%\lstset{commentstyle=\color{airforceblue}\itshape}
%\lstset{backgroundcolor=\color{beige}}